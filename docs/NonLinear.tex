 	\documentclass[12pt]{article}
\usepackage[polish]{babel}
\usepackage[utf8]{inputenc}
\usepackage[T1]{fontenc}
\usepackage{hyperref}
\usepackage{amsmath}

\title{Rozwiązywanie równań nieliniowych metodami połowienia, \textsl{regula falsi} i siecznych w arytmetyce przedziałowej}
\author{Mikołaj Rozwadowski}

\begin{document}
	\maketitle

	\section{Informacje ogólne}
		\subsection{Zastosowanie}
			Każda z niżej opisanych metod rozwiązuje w arytmetyce przedziałowej
			równanie $f(x) = 0$, gdzie $f$ jest dowolną funkcją rzeczywistą.
			Do znalezienia rozwiązania potrzebny jest początkowy przedział $[a, b]$,
			który jest potem sukcesywnie zawężany.

		\subsection{Sposób wczytania funkcji}
			Wymagana przez każdą metodę funkcja $f$ jest przekazywana jako wskaźnik
			do obiektu klasy \texttt{Function}. Można samemu napisać podklasę realizującą
			jakąkolwiek funkcję albo posłużyć się klasą \texttt{SOFunction},
			która umożliwia załadowanie jej z biblioteki dynamicznej \texttt{.so}
			pod systemami operacyjnymi z rodziny GNU/Linux.
			Sposób utworzenia i skompilowania kompatybilnej biblioteki opisany jest
			\href{https://github.com/hejmsdz/NonLinear/tree/master/functions}{na stronie projektu w serwisie GitHub}.

		\subsection{Identyfikatory nielokalne}
			W pliku nagłówkowym \texttt{common.h} znajdują się deklaracje
			pomocniczych funkcji i stałych, z których korzystają wszystkie metody:

			\begin{description}
				\item[\texttt{Function}] \hfill\\
					klasa udostępniająca metodę \texttt{interval evaluate(interval x)}.
				\item[\texttt{WRONG\_INTERVAL}] \hfill\\
					stała o wartości liczbowej 1; jest to wyjątek, którym funkcja sygnalizuje, że w podanym przedziale początkowym lewy koniec jest większy lub równy od prawego końca,
				\item[\texttt{NO\_REAL\_ROOTS}] \hfill\\
					stała o wartości liczbowej 2; jest to wyjątek, którym funkcja sygnalizuje brak pierwiastków rzeczywistych w podanym przedziale początkowym,
				\item[\texttt{check\_interval}] \hfill\\
					pomocnicza funkcja, która sprawdza warunki dla przedziału początkowego, a niespełnienie ich sygnalizuje zgłoszeniem jednego z powyższych wyjątków.
			\end{description}


	\section{Metoda połowienia przedziału}
		\subsection{Zastosowanie}
			Funkcja \texttt{Bisection} znajduje wartość pierwiastka równania
			$f(x) = 0$ metodą połowienia przedziału w arytmetyce przedziałowej.

		\subsection{Opis metody}
			Metoda połowienia wykorzystuje własność Darboux, która mówi,
			że jeśli dana jest funkcja ciągła $f$ i przedział rzeczywisty $[a, b]$ takie,
			że $f(a) = y_a$ i $f(b) = y_b$, to funkcja ta przyjmuje w przedziale $(a, b)$
			wszystkie wartości pośrednie między $y_a$ a $y_b$.
			W szczególności oznacza to, że jeśli taka funkcja ma
			na końcach przedziału różne znaki, to istnieje tam miejsce zerowe,
			czyli punkt $x_0 \in (a, b)$, dla którego $f(x_0) = 0$.

			Algorytm rozpoczyna pracę z danym przedziałem $[a, b]$,
			o którym wiadomo, że $f(a) \cdot f(b) < 0$.
			W każdej iteracji następuje wyznaczenie środka przedziału $m = \frac{a+b}{2}$
			i podział go na dwie połowy $[a, m]$ i $[m, b]$.
			Jeżeli $f(a) \cdot f(m) < 0$, to przeszukiwanie jest kontynuowane
			w przedziale $[a, m]$, w przeciwnym razie $[m, b]$.

			Pętla powtarza się aż do momentu, w którym środek przedziału $m$ będzie miejscem zerowym funkcji.
			Ponieważ w praktyce może to jednak nie nastąpić, do przerwanie algorytmu dojdzie też,
			gdy szerokość przedziału będzie mniejsza niż zadana tolerancja $\epsilon$
			lub po wykonaniu określonej liczby iteracji.
			W takim wypadku wynikiem będzie najwęższy uzyskany przedział zawierający pierwiastek.

		\subsection{Wywołanie funkcji}
			\texttt{Bisection(a, b, func, tolerance, iterations, reached)}

		\subsection{Dane}
			\begin{description}
				\item[\texttt{a}, \texttt{b}] \hfill\\ lewy i prawy koniec przedziału zawierającego pierwiastek,
				\item[\texttt{func}] \hfill\\ funkcja, której miesce zerowe należy znależć,
				\item[\texttt{tolerance}] \hfill\\ największa akceptowalna szerokość przedziału wynikowego,
				\item[\texttt{iterations}] \hfill\\ maksymalna liczba iteracji.
			\end{description}

		\subsection{Wyniki}
			\begin{description}
				\item[\texttt{Bisection(a, b, func, tolerance, iterations, reached)}] \hfill\\
					przedział zawierający miejsce zerowe
			\end{description}

		\subsection{Inne parametry}
			\begin{description}
				\item[\texttt{reached}] \hfill\\ określa, czy w \texttt{iterations} iteracjach udało się zmieścić w wymaganej tolerancji.
			\end{description}

		\subsection{Typy parametrów}
			\texttt{\textbf{interval} a, \textbf{interval} b, \textbf{Function*} func, \textbf{long double} tolerance, \textbf{int} iterations, \textbf{bool\&} reached}

		\subsection{Identyfikatory nielokalne}
			\begin{description}
				\item[\texttt{NOT\_ENOUGH\_ITERATIONS}] \hfill\\
					stała o wartości liczbowej 3; jest to wyjątek, którym funkcja sygnalizuje, że parametr \texttt{iterations} nie jest liczbą dodatnią.
			\end{description}

		\subsection{Przykłady}
			\begin{tabular}{|r|c|c|c|c||c|}
				\hline
				Równanie & a & b & liczba iteracji & $\epsilon$ & $x_0$ \\\hline

				$x^2 - 2 = 0$ & 1 & 2 & 60 & $10^{-16}$ & [1.41421356237309503, 1.41421356237309509] \\\hline
				$xe^{\sqrt{x+1}} - 1 = 0$ & -1 & 1 & 60 & $10^{-16}$ & [3.17347582146508266, 3.17347582146508323] \\\hline
				$\sin{x} \cdot (\sin{x} + \frac{1}{2}) - \frac{1}{2} = 0$ & 0 & 1 & 60 & $10^{-16}$ & [5.2359877559829803e-01, 5.2359877559829810e-01] \\\hline

			\end{tabular}



	\section{Metoda \textsl{regula falsi}}
		\subsection{Zastosowanie}
		\subsection{Opis metody}
		\subsection{Wywołanie funkcji}
		\subsection{Dane}
		\subsection{Wyniki}
		\subsection{Inne parametry}
		\subsection{Typy parametrów}
		\subsection{Identyfikatory nielokalne}
		\subsection{Przykłady}

	\section{Metoda siecznych}
		\subsection{Zastosowanie}
		\subsection{Opis metody}
		\subsection{Wywołanie funkcji}
		\subsection{Dane}
		\subsection{Wyniki}
		\subsection{Inne parametry}
		\subsection{Typy parametrów}
		\subsection{Identyfikatory nielokalne}
		\subsection{Przykłady}

	\section{Bibliografia}
		\begin{itemize}
			\item A. Marciniak, D. Gregulec, J. Kaczmarek: \textsl{Podstawowe procedury numeryczne w języku Turbo Pascal}. Wydawnictwo Nakom, Poznań, 2000 r.
		\end{itemize}

\end{document}
