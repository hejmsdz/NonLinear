\documentclass[12pt]{article}
\usepackage[polish]{babel}
\usepackage[utf8]{inputenc}
\usepackage[T1]{fontenc}
\usepackage{hyperref}
\usepackage{amsmath}
\usepackage{minted}

\setminted{breaklines}
\setminted{linenos=true}
\setminted{frame=leftline}

\title{Rozwiązywanie równań nieliniowych metodami połowienia, \textsl{regula falsi} i siecznych w arytmetyce przedziałowej}
\author{Mikołaj Rozwadowski}

\begin{document}
	\maketitle

	\section{Informacje ogólne}
		\subsection{Zastosowanie}
			Każda z niżej opisanych metod rozwiązuje w arytmetyce przedziałowej
			równanie $f(x) = 0$, gdzie $f$ jest dowolną funkcją rzeczywistą.
			Do znalezienia rozwiązania potrzebny jest początkowy przedział $[a, b]$,
			który jest potem sukcesywnie zawężany.

		\subsection{Sposób wczytania funkcji}
			Wymagana przez każdą metodę funkcja $f$ jest przekazywana jako wskaźnik
			do obiektu klasy \texttt{Function}. Można samemu napisać podklasę realizującą
			jakąkolwiek funkcję albo posłużyć się klasą \texttt{SOFunction},
			która umożliwia załadowanie jej z biblioteki dynamicznej \texttt{.so}
			pod systemami operacyjnymi z rodziny GNU/Linux.
			Sposób utworzenia i skompilowania kompatybilnej biblioteki opisany jest
			\href{https://github.com/hejmsdz/NonLinear/tree/master/functions}{na stronie projektu w serwisie GitHub}.

		\subsection{Identyfikatory nielokalne}\label{nonlocal}
			W pliku nagłówkowym \texttt{common.h} znajdują się deklaracje
			pomocniczych funkcji i stałych, z których korzystają wszystkie metody:

			\begin{description}
				\item[\texttt{Function}] \hfill\\
					klasa udostępniająca metodę \texttt{interval evaluate(interval x)}.
				\item[\texttt{WRONG\_INTERVAL}] \hfill\\
					stała o wartości liczbowej 1; jest to wyjątek, którym funkcja sygnalizuje, że w podanym przedziale początkowym lewy koniec jest większy lub równy od prawego końca,
				\item[\texttt{NO\_REAL\_ROOTS}] \hfill\\
					stała o wartości liczbowej 2; jest to wyjątek, którym funkcja sygnalizuje niespełnienie warunku $f(a) \cdot f(b) < 0$,
				\item[\texttt{check\_interval}] \hfill\\
					pomocnicza funkcja, która sprawdza warunki dla przedziału początkowego, a niespełnienie ich sygnalizuje zgłoszeniem jednego z powyższych wyjątków,
        \item[\texttt{sgn}] \hfill\\
					pomocnicza funkcja wyznaczająca znak przedziału, przy czym jeśli porównanie nie jest jednoznaczne, to przyjmowany jest znak 0.
			\end{description}


	\section{Metoda połowienia przedziału}
		\subsection{Zastosowanie}
			Funkcja \texttt{Bisection} znajduje wartość pierwiastka równania
			$f(x) = 0$ metodą połowienia przedziału w arytmetyce przedziałowej.

		\subsection{Opis metody}
			Metoda połowienia wykorzystuje własność Darboux, która mówi,
			że jeśli dana jest funkcja ciągła $f$ i przedział rzeczywisty $[a, b]$ takie,
			że $f(a) = y_a$ i $f(b) = y_b$, to funkcja ta przyjmuje w przedziale $(a, b)$
			wszystkie wartości pośrednie między $y_a$ a $y_b$.
			W szczególności oznacza to, że jeśli taka funkcja ma
			na końcach przedziału różne znaki, to istnieje tam miejsce zerowe,
			czyli punkt $x_0 \in (a, b)$, dla którego $f(x_0) = 0$.

			Algorytm rozpoczyna pracę z danym przedziałem $[a, b]$,
			o którym wiadomo, że $f(a) \cdot f(b) < 0$.
			W każdej iteracji następuje wyznaczenie środka przedziału $m = \frac{a+b}{2}$
			i podział go na dwie połowy $[a, m]$ i $[m, b]$.
			Jeżeli $f(a) \cdot f(m) < 0$, to przeszukiwanie jest kontynuowane
			w przedziale $[a, m]$, w przeciwnym razie $[m, b]$.

			Pętla powtarza się aż do momentu, w którym środek przedziału $m$ będzie miejscem zerowym funkcji.
			Ponieważ w praktyce może to jednak nie nastąpić, do przerwania algorytmu dojdzie też,
			gdy szerokość przedziału będzie mniejsza niż zadana tolerancja $\epsilon$
			lub po wykonaniu określonej liczby iteracji.
			W takim wypadku wynikiem będzie najwęższy uzyskany przedział zawierający pierwiastek.

		\subsection{Wywołanie funkcji}
			\texttt{Bisection(a, b, func, tolerance, iterations, reached)}

		\subsection{Dane}
			\begin{description}
				\item[\texttt{a}, \texttt{b}] \hfill\\ lewy i prawy koniec przedziału zawierającego pierwiastek,
				\item[\texttt{func}] \hfill\\ funkcja, której miesce zerowe należy znależć,
				\item[\texttt{tolerance}] \hfill\\ największa akceptowalna szerokość przedziału wynikowego,
				\item[\texttt{iterations}] \hfill\\ maksymalna liczba iteracji.
			\end{description}

		\subsection{Wyniki}
			\begin{description}
				\item[\texttt{Bisection(a, b, func, tolerance, iterations, reached)}] \hfill\\
					przedział zawierający miejsce zerowe
			\end{description}

		\subsection{Inne parametry}
			\begin{description}
				\item[\texttt{reached}] \hfill\\ określa, czy w \texttt{iterations} iteracjach udało się zmieścić w wymaganej tolerancji.
			\end{description}

      Jeśli początkowy przedział $[a, b]$ nie będzie spełniać wymagań, to funkcja wyrzuci wyjątek (zob. \ref{nonlocal}).
      Również za pomocą wyjątku odrzucone zostanie podanie liczby iteracji mniejszej lub równej 0.

		\subsection{Typy parametrów}
			\texttt{\textbf{interval} a, \textbf{interval} b, \textbf{Function*} func, \textbf{long double} tolerance, \textbf{int} iterations, \textbf{bool\&} reached}

		\subsection{Identyfikatory nielokalne}
      Oprócz wymienionych w punkcie \ref{nonlocal} plik źródłowy definiuje następujące identyfikatory nielokalne:

      \begin{description}
				\item[\texttt{NOT\_ENOUGH\_ITERATIONS}] \hfill\\
					stała o wartości liczbowej 3; jest to wyjątek, którym funkcja sygnalizuje, że parametr \texttt{iterations} nie jest liczbą dodatnią.
			\end{description}

		\subsection{Kod źródłowy}
		  \inputminted[firstline=3, lastline=41]{c++}{../solvers/bisection.cpp}

		\subsection{Przykłady}
			\begin{tabular}{|r|c|c||c|}
				\hline
				Równanie & a & b & $x_0$ \\\hline

				$x^2 - 2 = 0$ & 1 & 2 & [1.4142135623730950, 1.4142135623730951] \\\hline
				$xe^{\sqrt{x+1}} - 1 = 0$ & -1 & 1 & [3.17347582146508266, 3.17347582146508323] \\\hline
				$\sin{x} \cdot (\sin{x} + \frac{1}{2}) - \frac{1}{2} = 0$ & [0.4, 0.5] & 1 & [5.2359877559829801e-01, 5.2359877559829809e-01] \\\hline

			\end{tabular}

			We wszystkich przykładach przyjęto liczbę iteracji = 60 i $\epsilon = 10^{-16}$.


	\section{Metoda \textsl{regula falsi}}
		\subsection{Zastosowanie}
  		Funkcja \texttt{RegulaFalsi} znajduje wartość pierwiastka równania
  		$f(x) = 0$ metodą \textsl{regula falsi} w arytmetyce przedziałowej.

		\subsection{Opis metody}
  		Metoda \textsl{regula falsi} opiera się na założeniu, że każdą funkcję
  		można w odpowiednio małym zakresie argumentów traktować jak funkcję liniową.
      Choć z matematycznego punktu widzenia jest to nieprawda (stąd nazwa --
      \textsl{regula falsi} to po łacinie fałszywa zasada albo fałszywa prosta),
      to obliczanie kolejnych miejsc zerowych tak, jakby funkcja
      rzeczywiście była na tym odcinku liniowa, daje coraz lepsze przybliżenia
      prawdziwego pierwiastka.

      W każdej iteracji algorytmu wyznaczany jest punkt przecięcia prostej
      przechodzącej przez punkty $(a, f(a))$ i $(b, f(b))$ z osią $X$:

      \begin{equation*}
        x = b - f(b) \cdot \frac{b - a}{f(b) - f(a)},
      \end{equation*}

      a następnie uzyskany punkt zastępuje lewy lub prawy koniec przedziału początkowego.
      Pętla trwa dopóki spełniony jest warunek $a > x > b$.

    \subsection{Wywołanie funkcji}
  		\texttt{RegulaFalsi(a, b, func)}

  	\subsection{Dane}
  		\begin{description}
				\item[\texttt{a}, \texttt{b}] \hfill\\ lewy i prawy koniec przedziału zawierającego pierwiastek,
				\item[\texttt{func}] \hfill\\ funkcja, której miesce zerowe należy znależć.
			\end{description}

		\subsection{Wyniki}
			\begin{description}
				\item[\texttt{RegulaFalsi(a, b, func)}] \hfill\\
					przedział zawierający miejsce zerowe
			\end{description}

		\subsection{Inne parametry}
			Brak. Jeśli początkowy przedział $[a, b]$ nie będzie spełniać wymagań, to funkcja wyrzuci wyjątek (zob. \ref{nonlocal}).

		\subsection{Typy parametrów}
			\texttt{\textbf{interval} a, \textbf{interval} b, \textbf{Function*} func}

		\subsection{Identyfikatory nielokalne}
      Brak, nie licząc wymienionych w punkcie \ref{nonlocal}.

		\subsection{Kod źródłowy}
			\inputminted[firstline=3, lastline=36]{c++}{../solvers/regulafalsi.cpp}

		\subsection{Przykłady}
			\begin{tabular}{|r|c|c||c|}
				\hline
				Równanie & a & b & $x_0$ \\\hline

				$x^2 - 2 = 0$ & 1 & 2 & [1.4142135623730950, 1.4142135623734952]\\\hline
				$x^2 - 2 = 0$ & 0.3 & [1.5, 1.6] & [2.9999999999999999e-01, 1.6000000000000001]\\\hline
				$\sin{x} \cdot (\sin{x} + \frac{1}{2}) - \frac{1}{2} = 0$ & 0.1 & 1 & [5.2359877559652552e-01, 1] \\\hline

			\end{tabular}

	\section{Metoda siecznych}
		\subsection{Zastosowanie}
    Funkcja \texttt{Secant} znajduje wartość pierwiastka równania
    $f(x) = 0$ metodą siecznych w arytmetyce przedziałowej.

		\subsection{Opis metody}
			Podobnie jak \textsl{regula falsi}, metoda siecznych również opiera się na interpolacji liniowej.
			Algorytm konstruuje zbieżny do dokładnej wartości pierwiastka ciąg~przybliżeń $(x_i)$ według rekurencyjnego wzoru:
			\begin{equation*}
				x_i = x_{i-1} - f(x_{i-1}) \frac{x_{i-1} - x_{i-2}}{f(x_{i-1}) - f(x_{i-2})}.
			\end{equation*}

			Pierwsze przybliżenia wyznaczane są na podstawie końców przedziału początkowego jako:
			$x_1 = a + h$, $x_2 = b - h$, $h = 0,179372 \cdot (b-a)$, przy czym jeśli nie jest spełniony warunek
			$|f(x_1)| \geq |f(x_2)|$, to wartości te są na początku zamieniane miejscami.

		\subsection{Wywołanie funkcji}
      \texttt{Secant(a, b, func)}

		\subsection{Dane}
      \begin{description}
        \item[\texttt{a}, \texttt{b}] \hfill\\ lewy i prawy koniec przedziału zawierającego pierwiastek,
        \item[\texttt{func}] \hfill\\ funkcja, której miesce zerowe należy znależć.
      \end{description}

		\subsection{Wyniki}
      \begin{description}
        \item[\texttt{Secant(a, b, func)}] \hfill\\
          przedział zawierający miejsce zerowe
      \end{description}

		\subsection{Inne parametry}
      Brak. Jeśli początkowy przedział $[a, b]$ nie będzie spełniać wymagań, to funkcja wyrzuci wyjątek (zob. \ref{nonlocal}).

		\subsection{Typy parametrów}
      \texttt{\textbf{interval} a, \textbf{interval} b, \textbf{Function*} func}

		\subsection{Identyfikatory nielokalne}
      Brak, nie licząc wymienionych w punkcie \ref{nonlocal}.

		\subsection{Kod źródłowy}
			\inputminted[firstline=6, lastline=37]{c++}{../solvers/secant.cpp}

		\subsection{Przykłady}
			\begin{tabular}{|r|c|c||c|}
				\hline
				Równanie & a & b & $x_0$ \\\hline

				$x^2 - 2 = 0$ & 1 & 2 & [1.4142135623730950, 1.4142135623734952]\\\hline
				$xe^{\sqrt{x+1}} - 1 = 0$ & -1 & 1 & [3.1734757786211827e-01, 3.1734758214652323e-01]\\\hline
				$xe^{\sqrt{x+1}} - 1 = 0$ & [-0.5, -0.4] & [0.2, 0.4] & [-3.9237680000000000e-01, 1.2677667075395474]\\\hline

			\end{tabular}

	\section{Bibliografia}
		\begin{itemize}
		  \item A. Marciniak, D. Gregulec, J. Kaczmarek: \textsl{Podstawowe procedury numeryczne w języku Turbo Pascal}. NAKOM, Poznań, 2000~r.
		\end{itemize}

\end{document}
