\documentclass[12pt]{article}
\usepackage[polish]{babel}
\usepackage[utf8]{inputenc}
\usepackage[T1]{fontenc}
\usepackage{hyperref}
\usepackage{amsmath}

\title{Rozwiązywanie równań nieliniowych metodami połowienia, \textsl{regula falsi} i siecznych w arytmetyce przedziałowej}
\author{Mikołaj Rozwadowski}

\begin{document}
	\maketitle

	\section{Metoda połowienia przedziału}
		\subsection{Zastosowanie}
		\subsection{Opis metody}
		\subsection{Wywołanie funkcji}
		\subsection{Dane}
		\subsection{Wyniki}
		\subsection{Inne parametry}
		\subsection{Typy parametrów}
		\subsection{Identyfikatory nielokalne}
		\subsection{Przykłady}

	\section{Metoda \textsl{regula falsi}}
		\subsection{Zastosowanie}
		\subsection{Opis metody}
		\subsection{Wywołanie funkcji}
		\subsection{Dane}
		\subsection{Wyniki}
		\subsection{Inne parametry}
		\subsection{Typy parametrów}
		\subsection{Identyfikatory nielokalne}
		\subsection{Przykłady}

	\section{Metoda siecznych}
		\subsection{Zastosowanie}
		\subsection{Opis metody}
		\subsection{Wywołanie funkcji}
		\subsection{Dane}
		\subsection{Wyniki}
		\subsection{Inne parametry}
		\subsection{Typy parametrów}
		\subsection{Identyfikatory nielokalne}
		\subsection{Przykłady}

	\section{Bibliografia}
		\begin{itemize}
			\item A. Marciniak, D. Gregulec, J. Kaczmarek: \textsl{Podstawowe procedury numeryczne w języku Turbo Pascal}. Wydawnictwo Nakom, Poznań, 2000 r.
		\end{itemize}

\end{document}
